\section*{p-adische Struktur als Modell hierarchischer Demokratie}

Eine $5$-adische Zahl besitzt die Darstellung
\[
x = a_0 + a_1\cdot 5 + a_2\cdot 5^2 + a_3\cdot 5^3 + \dots,
\qquad a_i \in \{0,1,2,3,4\}.
\]

Wir interpretieren die Koeffizienten $a_i$ als Entscheidungen auf
hierarchischen politischen Ebenen:
\[
\begin{aligned}
a_0 &\;\longrightarrow\; \text{Dorfebene},\\
a_1 &\;\longrightarrow\; \text{Bezirksebene},\\
a_2 &\;\longrightarrow\; \text{Bundesland},\\
a_3 &\;\longrightarrow\; \text{Nationale Ebene},\\
a_4 &\;\longrightarrow\; \text{Supranationale Ebene},\;\text{etc.}
\end{aligned}
\]

Höhere Potenzen $5^k$ entsprechen dabei höheren Entscheidungsebenen.
Die p-adische Nähe zweier Entscheidungen $x$ und $y$ wird durch die
$5$-adische Norm
\[
|x-y|_5 = 5^{-k}
\]
bestimmt, wobei $k$ die größte Ebene ist, auf der $x$ und $y$
identische Entscheidungen teilen.

\medskip
\noindent
\textbf{Interpretation:}
Zwei politische Positionen gelten als ``nah'', wenn sie auf vielen
höheren Ebenen übereinstimmen. Unterschiede auf unteren Ebenen
(Dorf/Bezirk) sind strukturell klein, solange Konsens auf höheren
Ebenen besteht. Bricht der Konsens auf hoher Ebene, steigt der Abstand
sprunghaft --- analog zur p-adischen Metrik.

\[
\text{Stabilität} \;\propto\; \text{Tiefe der gemeinsamen Entscheidungshierarchie}.
\]

