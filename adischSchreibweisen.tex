\documentclass[12pt,a4paper]{article}
\usepackage[utf8]{inputenc}
\usepackage[T1]{fontenc}
\usepackage[ngerman]{babel}
\usepackage{amsmath, amssymb, amsthm}
\usepackage{mathtools}
\usepackage{hyperref}

\title{p-adische Zahlen — strukturelle Beschreibung ohne Zahlenschreibweise}
\author{}
\date{}

\begin{document}
\maketitle

\section*{Kernidee ohne Zahlenschreibweise}

Eine p-adische Zahl ist ein Äquivalenz- und Vollständigkeitsobjekt: 
Man startet mit den rationalen Zahlen $\mathbb{Q}$.
Man misst Größen nicht nach ihrer absoluten Größe, sondern nach ihrer
Teilbarkeit durch $p$. Daraus entsteht eine Metrik, und $\mathbb{Q}$
wird komplettiert — das Resultat ist $\mathbb{Q}_p$.

\subsection*{1. p-adische Bewertung}

Für eine rationale Zahl $x \ne 0$ gibt es eine Darstellung
\[
x = p^k \frac{a}{b}, \qquad \gcd(a,p)=\gcd(b,p)=1,
\]
und man definiert
\[
v_p(x)=k.
\]

\subsection*{2. p-adische Norm}

\[
|x|_p = p^{-v_p(x)}.
\]

\subsection*{3. p-adische Metrik}

\[
d_p(x,y)=|x-y|_p.
\]

\subsection*{4. Vervollständigung}

\[
\mathbb{Q}_p = \text{Komplettierung von } (\mathbb{Q},d_p).
\]
Damit ist eine p-adische Zahl kein Ziffern-String, sondern ein Grenzwert
von Cauchy-Folgen bezüglich $d_p$.

\section*{Alternative, gleichwertige Beschreibungen}

\subsection*{(1) Als Inverses Limes-Objekt}

\[
\mathbb{Z}_p = \varprojlim \mathbb{Z} / p^n \mathbb{Z}.
\]
Eine p-adische ganze Zahl ist eine konsistente Familie von Restklassen
mod $p, p^2, p^3, \dots$

\subsection*{(2) Als kompakter topologischer Ring}

\begin{itemize}
\item $\mathbb{Z}_p$ ist kompakt, vollständig und total unzusammenhängend.
\item $\mathbb{Q}_p = \mathbb{Z}_p[p^{-1}]$.
\end{itemize}

\subsection*{(3) Als lokaler Körper}

\begin{itemize}
\item Maximales Ideal: $p \mathbb{Z}_p$
\item Restklassenkörper: $\mathbb{F}_p$
\item Diskreter Bewertungsring mit Uniformisator $p$
\end{itemize}

\subsection*{(4) Archimedisch vs.\ nicht-archimedisch}

\begin{align*}
\text{Reelle Zahlen} & = \text{Komplettierung nach der archimedischen Norm}, \\
\text{p-adische Zahlen} & = \text{Komplettierung nach einer nicht-archimedischen Norm}.
\end{align*}

Die Dreiecksungleichung wird ultrametrisch:
\[
|x+y|_p \le \max(|x|_p,|y|_p).
\]

\section*{Intuitive Kurzform}

\begin{itemize}
\item Reelle Zahlen messen „Größe im Unendlichen“.
\item p-adische Zahlen messen „Tiefe der Teilbarkeit durch $p$“.
\item Zwei Zahlen sind p-adisch nahe, wenn ihre Differenz stark durch $p$ teilbar ist.
\end{itemize}

\end{document}
    
