\section*{Beispiele zur Zahl--Vektor--Korrespondenz}

\subsection*{Beispiel 1: Zahl $\rightarrow$ Vektor in Basis $5$}

Gegeben sei
\[
x = 347.
\]
Wir bestimmen die Ziffern $a_i$ durch sukzessive Division mit Rest.

\[
a_0 = 347 \bmod 5 = 2,
\qquad
\left\lfloor \frac{347}{5}\right\rfloor = 69,
\]
\[
a_1 = 69 \bmod 5 = 4,
\qquad
\left\lfloor \frac{69}{5}\right\rfloor = 13,
\]
\[
a_2 = 13 \bmod 5 = 3,
\qquad
\left\lfloor \frac{13}{5}\right\rfloor = 2,
\]
\[
a_3 = 2 \bmod 5 = 2,
\qquad
\left\lfloor \frac{2}{5}\right\rfloor = 0.
\]

Damit endet die Expansion und es gilt
\[
\Phi_5(347) = (a_0,a_1,a_2,a_3) = (2,4,3,2)_5.
\]

\paragraph{Verifikation.}
\[
2 + 4\cdot 5 + 3\cdot 5^2 + 2\cdot 5^3
= 2 + 20 + 75 + 250 = 347.
\]
Also stimmt die Darstellung.


\subsection*{Beispiel 2: Vektor $\rightarrow$ Zahl in Basis $7$}

Gegeben sei der Koeffizientenvektor
\[
(a_0,a_1,a_2,a_3) = (3,5,2,1)_7.
\]

Die Syntheseabbildung lautet
\[
x = \Psi_7(3,5,2,1)
   = 3 + 5\cdot 7 + 2\cdot 7^2 + 1\cdot 7^3.
\]

Wir berechnen die Terme:
\[
5\cdot 7 = 35,\qquad
2\cdot 7^2 = 2\cdot 49 = 98,\qquad
1\cdot 7^3 = 343.
\]

Somit
\[
x = 3 + 35 + 98 + 343 = 479.
\]

\paragraph{Verifikation.}
Die Rückabbildung ergibt
\[
\Phi_7(479) = (3,5,2,1)_7,
\]
also sind $\Phi_7$ und $\Psi_7$ hier invers zueinander.

