\documentclass[11pt]{article}
\usepackage{amsmath,amssymb,mathtools}
\title{Zweiseitige $p$-adische Darstellung perfektoider Zahlen}
\date{}
\begin{document}
\maketitle

\section*{Zweiseitige $p$-adische Darstellung}

Eine **perfektoide Zahl** kann vollständig durch eine **zweiseitige $p$-adische Reihe** dargestellt werden:

\[
x
=
\underbrace{\cdots + a_2 p^2 + a_1 p + a_0}_{\text{$p$-adische Richtung nach links}}
\;\Big|\;
\underbrace{a_{-1} p^{-1} + a_{-2} p^{-2} + \cdots}_{\text{negative / Laurent-Richtung nach rechts}}
\]

oder formal als **zweiseitige Laurent-Reihe**:

\[
x = \sum_{n=-\infty}^{+\infty} a_n p^n,
\qquad a_n \in A,
\]

wobei \(A\) ein perfektoider Ring ist.

\bigskip

\section*{Begründung}

Diese Darstellung enthält **alle Informationen**, die zur eindeutigen Beschreibung einer perfektoiden Zahl nötig sind:

\begin{itemize}
    \item Die linke Seite ($n\ge 0$) erfasst die **klassische $p$-adische Expansion**.
    \item Die rechte Seite ($n<0$) erfasst die **negativen Potenzen** bzw. eine Laurent-ähnliche Richtung.
    \item Über beide Richtungen werden **beliebige perfektoide Zahlen eindeutig kodiert**, ohne dass zusätzliche Marker, Operatoren oder Informationen nötig sind.
\end{itemize}

\bigskip

\section*{Rechenoperationen}

Addition, Subtraktion, Multiplikation und Division können direkt über die Koeffizienten durchgeführt werden:

\[
\sum_{n} a_n p^n \;\pm\; \sum_{n} b_n p^n = \sum_{n} (a_n \pm b_n) p^n,
\]

\[
\left(\sum_{i} a_i p^i\right) \cdot \left(\sum_{j} b_j p^j\right)
=
\sum_{n} \left(\sum_{i+j=n} a_i b_j\right) p^n,
\]

\[
\frac{1}{\sum_{n\ge 0} a_n p^n} = \sum_{k=0}^{\infty} (-1)^k \left(\sum_{n\ge1} a_n p^n\right)^k.
\]

Damit sind alle **Grundoperationen in perfektoiden Zahlen** konsistent und vollständig über diese Darstellung ausführbar.

\bigskip

\section*{Schlussfolgerung}

\begin{quote}
Die zweiseitige $p$-adische Darstellung mit zwei Kommas (linke $p$-adische Richtung, rechte negative Richtung) ist **hinreichend**, um eine perfektoide Zahl vollständig zu repräsentieren.  
Es werden keine weiteren Informationen benötigt.
\end{quote}

\end{document}
