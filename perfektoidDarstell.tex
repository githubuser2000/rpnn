\documentclass[11pt]{article}
\usepackage{amsmath,amssymb,mathtools}

\title{Zahlendarstellung mit zwei Unendlichkeitsrichtungen
als perfektoide $p$-adische Struktur}
\date{}
\begin{document}
\maketitle

\section*{Idee der Darstellung}

Wir betrachten eine Darstellung
\[
x
=
\cdots + a_2 p^2 + a_1 p + a_0
\;\;\Big|\;\;
a_{-1}p^{-1} + a_{-2}p^{-2} + \cdots ,
\]
wobei der linke Teil eine $p$-adische Richtung (hohe Potenzen $p^n$)
und der rechte Teil eine Richtung negativer Potenzen beschreibt.
Formal schreiben wir
\[
x
=
\sum_{n=-\infty}^{+\infty} a_n p^n,
\qquad
a_n \in A,
\]
also eine \emph{zweiseitige $p$-adische Laurent-Reihe}.

\bigskip

\section*{Definition}

Sei $A$ ein perfektoider Ring.
Wir definieren den Ring
\[
\mathcal{L}_p(A)
=
\left\{
\sum_{n=-\infty}^{+\infty} a_n p^n
\;\middle|\;
a_n \in A,\;
a_n \to 0 \text{ für } n\to +\infty
\right\},
\]
mit
\[
p\text{-adischer Konvergenz nach links}
\quad\text{und}\quad
\text{Laurent-Struktur nach rechts}.
\]

Addition und Multiplikation erfolgen formal koeffizientenweise:
\[
\left(\sum a_n p^n\right)
+
\left(\sum b_n p^n\right)
=
\sum (a_n+b_n)p^n,
\]
\[
\left(\sum a_i p^i\right)
\left(\sum b_j p^j\right)
=
\sum_{n\in\mathbb{Z}}
\left(\sum_{i+j=n} a_i b_j\right)p^n.
\]

\bigskip

\section*{Zwei ``Kommas'' als Strukturmarker}

Wir interpretieren die Darstellung symbolisch als
\[
\underbrace{
\cdots + a_2p^2 + a_1p + a_0
}_{\text{$p$-adische Richtung}}
\;\;\Vert\;\;
\underbrace{
a_{-1}p^{-1}+a_{-2}p^{-2}+\cdots
}_{\text{negative / archimedische Richtung}}.
\]

Damit wird eine Zahl durch zwei Unendlichkeitsrichtungen
zwischen zwei Strukturmarkern beschrieben.

\bigskip

\section*{Bezug zu perfektoiden Körpern}

Aus einem perfektoiden Ring $A$ erhält man den Tilt
\[
A^\flat = \varprojlim_{x\mapsto x^p} A/p.
\]
Die zugehörigen Fontaine-Körper tragen Darstellungen der Form
\[
\sum_{n\gg -\infty} a_n t^n,
\]
also Laurent-Reihen mit positiver und negativer Richtung,
analytisch vergleichbar zur obigen Struktur.
Insbesondere liegt
\[
\mathcal{L}_p(A)
\]
konzeptionell nahe an perfektoiden Erweiterungen wie
\[
\mathbf{B}_{\mathrm{dR}}.
\]

\bigskip

\section*{Interpretation}

Die Darstellung mit zwei ``Kommas'' beschreibt somit ein Element,
das gleichzeitig

\begin{itemize}
\item $p$-adisch nach oben vervollständigt ist und
\item eine zweite, nach unten offene Laurent-Richtung besitzt.
\end{itemize}

Dies realisiert eine kombinierte
\[
\text{$p$-adische} \;+\; \text{Laurent-/archimedische Skala}
\]
als perfektoid motivierte Zahlstruktur.

\end{document}
