\section*{Formales Modell einer hierarchischen Demokratie auf p-adischer Struktur}

\subsection*{1. Entscheidungsbaum}

Sei $T$ ein geordneter, wurzelbehafteter Baum mit Ebenen
\[
L_0, L_1, L_2, \dots
\]
die den politischen Hierarchieebenen entsprechen:
\[
L_0 = \text{Dorf},\quad
L_1 = \text{Bezirk},\quad
L_2 = \text{Bundesland},\quad
L_3 = \text{Nation},\ \dots
\]

Jeder Knoten $v \in L_k$ besitzt eine endliche Menge von Kindern
$\mathrm{Child}(v) \subseteq L_{k-1}$.

Eine \emph{Entscheidungskonfiguration} ist ein unendlicher Pfad
\[
x = (x_0,x_1,x_2,\dots)
\]
mit $x_k \in L_k$ und $x_{k-1} \in \mathrm{Child}(x_k)$.

\subsection*{2. Kodierung als $p$-adische Expansion}

Jedem Knoten $x_k$ wird ein lokaler Entscheidungswert
\[
a_k(x) \in \{0,1,\dots,p-1\}
\]
zugeordnet. Damit definieren wir die Abbildung
\[
\Phi(x)
  = \sum_{k=0}^{\infty} a_k(x)\,p^k
\]
als p-adische Repräsentation der Entscheidung $x$.

\subsection*{3. Konsenstiefe}

Für zwei Entscheidungen $x,y$ sei
\[
k(x,y)
= \max\{\,k \mid x_j = y_j \text{ für alle } j \ge k\,\}
\]
die größte gemeinsame Hierarchieebene
(\emph{Tiefe des gemeinsamen politischen Konsenses}).

\subsection*{4. p-adische Entscheidungsmetrik}

Die Distanz wird definiert als
\[
d(x,y)
= p^{-k(x,y)}.
\]

Damit gilt:
\[
d(x,y) \text{ ist klein} \;\Longleftrightarrow\;
\text{tiefer Konsens auf höheren Ebenen.}
\]

\subsection*{5. Eigenschaften}

\begin{itemize}
\item $d$ ist eine Ultrametrik:
\[
d(x,z) \le \max\{d(x,y),d(y,z)\}.
\]
\item Konflikte auf unteren Ebenen sind strukturell klein,
solange höhere Ebenen übereinstimmen.
\item Systeminstabilität entsteht, wenn Konsens auf hoher Ebene bricht.
\end{itemize}

\subsection*{6. Interpretation}

\[
\text{Stabilität}(x,y)
\;\sim\;
k(x,y)
\quad\text{(gemeinsame Entscheidungstiefe)}.
\]

