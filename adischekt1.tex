\documentclass[a4paper,12pt]{article}
\usepackage[utf8]{inputenc}
\usepackage{amsmath, amssymb}
\usepackage{geometry}
\usepackage{hyperref}
\usepackage{xcolor}
\usepackage{listings}
\geometry{margin=1in}

\title{Perfektoide Darstellung einer 3-adischen Zahl}
\author{Alexander Kern \& ChatGPT}
\date{\today}

\begin{document}

\maketitle

\section*{Gesprächszusammenfassung}
Hier dokumentieren wir die Konversation und die Umsetzung einer 3-adischen Zahl …210 in perfektoide ganze Zahlen.

\section{3-adische Zahl in Dezimal}
Die 3-adische Zahl …210 (Ziffern von rechts nach links: $a_0=0$, $a_1=1$, $a_2=2$) entspricht in Dezimal:

\[
x = 0 \cdot 3^0 + 1 \cdot 3^1 + 2 \cdot 3^2 = 21
\]

\section{Perfektoide Darstellung}
Jede Zahl kann in positive ganze Zahlen zerlegt werden. Beispiele für 21:

\[
[21], [18,3], [9,9,3], [1,1,\dots,1]
\]

\section{Python-Algorithmus}

\begin{lstlisting}[language=Python, basicstyle=\ttfamily\small, keywordstyle=\color{blue}]
# Python-Algorithmus für perfektoide Arrays

import numpy as np
import pandas as pd
import sympy as sp
import itertools
import math
import functools
import operator
import networkx as nx
import matplotlib.pyplot as plt
import seaborn as sns
from tqdm import tqdm
from rich import print
from more_itertools import distinct_permutations, powerset
from typing import List

def all_partitions(n: int) -> List[List[int]]:
    if n == 0:
        return [[0]]
    result = []
    for i in range(1, n + 1):
        for tail in all_partitions(n - i):
            if tail == [0]:
                result.append([i])
            else:
                result.append([i] + tail)
    return result

def perfektoid_arrays_3adic(digits: List[int]) -> List[List[int]]:
    result = [[0]]
    for i, d in enumerate(digits):
        power = 3**i
        new_result = []
        for partition in all_partitions(d):
            scaled = [x * power for x in partition]
            for r in result:
                new_result.append(r + scaled)
        result = new_result
    return result

digits = [0,1,2]
arrays = perfektoid_arrays_3adic(digits)
print(arrays[:20])
\end{lstlisting}

\section{Ergebnisse}
Die ersten 20 perfektoiden Arrays der 3-adischen Zahl …210:

\[
[0,1,18], [0,1,9,9], [0,1,9,5,4], \dots
\]

\end{document}

