\section*{Zahl--Vektor--Korrespondenz in Basis $n$}

Sei $n \in \mathbb{N}$ mit $n \ge 2$.
Wir definieren den Koeffizientenraum
\[
C_n = \{(a_0,a_1,\dots,a_k)\mid k\in\mathbb{N},\; a_i\in\{0,\dots,n-1\}\}.
\]

\subsection*{1. Abbildung Zahl $\rightarrow$ Vektor (Basisexpansion)}
Für $x\in\mathbb{N}$ definieren wir die Ziffern
\[
a_i = \left\lfloor \frac{x}{n^i}\right\rfloor \bmod n,
\qquad i\ge 0,
\]
so dass
\[
x = \sum_{i=0}^{k} a_i n^i
\]
für ein minimales $k$ gilt.
Die Abbildung
\[
\Phi_n:\ \mathbb{N}\to C_n,\qquad
\Phi_n(x)=(a_0,\dots,a_k)
\]
heißt \emph{Basisexpansion (Zahl $\rightarrow$ Vektor)}.

\subsection*{2. Abbildung Vektor $\rightarrow$ Zahl (Synthese)}
Sei $(a_0,\dots,a_k)\in C_n$.
Wir definieren
\[
\Psi_n(a_0,\dots,a_k)=\sum_{i=0}^{k} a_i n^i.
\]
Dies ist die \emph{Syntheseabbildung (Vektor $\rightarrow$ Zahl)}.

\subsection*{3. Isomorphie im endlichen Fall}
Für alle $x\in\mathbb{N}$ und alle $(a_0,\dots,a_k)\in C_n$ gilt
\[
\Psi_n(\Phi_n(x)) = x,
\qquad
\Phi_n(\Psi_n(a_0,\dots,a_k)) = (a_0,\dots,a_k).
\]
Damit sind $\Phi_n$ und $\Psi_n$ zueinander inverse Bijektionen\\
zwischen endlichen Basisdarstellungen und ganzen Zahlen.

\subsection*{4. p-adische Erweiterung}
Für eine Primzahl $p$ definieren wir den Raum unendlicher Koeffizientenfolgen
\[
\mathbb{Z}_p=\left\{
(a_0,a_1,a_2,\dots)\ \bigg|\ a_i\in\{0,\dots,p-1\}
\right\},
\]
mit der p-adischen Auswertung
\[
x=\sum_{i=0}^{\infty} a_i p^i
\]
und der ultrametrischen Norm
\[
|p^k u|_p = p^{-k},\qquad u\in\mathbb{Z},\; p\nmid u.
\]
Damit wird $\mathbb{Z}_p$ ein vollständiger normierter Raum
(ultrametrischer Vektorraum über $\mathbb{Z}_p$).
