\section*{Spieltheorie und Simulation auf p-adischer Entscheidungsstruktur}

\subsection*{1. Entscheidungsraum und Metrik}

Sei $X$ die Menge aller unendlichen Pfade in einem hierarchischen
Entscheidungsbaum
\[
x = (x_0,x_1,x_2,\dots),
\qquad x_k \in L_k,
\]
wobei $L_k$ die politische Ebene der Tiefe $k$ bezeichnet
(Dorf, Bezirk, Bundesland, Nation, \dots).

Die p-adische Konsenstiefe zweier Entscheidungen $x,y \in X$ ist
\[
k(x,y)
=
\max\{\,k \mid x_j = y_j \text{ für alle } j \ge k\,\}.
\]

Die zugehörige Ultrametrik lautet
\[
d(x,y)=p^{-k(x,y)}.
\]

\subsection*{2. Konfliktkosten nach Hierarchieebenen}

Seien $w_0 < w_1 < w_2 < \dots$ monoton wachsende
Gewichte politischer Konflikte.
Für zwei Entscheidungen definieren wir
\[
C(x,y)
=
\sum_{k\ge 0}
w_k \,\mathbf{1}[x_k \neq y_k],
\]
wobei Konflikte auf höheren Ebenen stärker gewichtet werden.

\subsection*{3. Nutzenfunktion der Spieler}

Es gebe $N$ Spieler mit Entscheidungen
$x^{(1)},\dots,x^{(N)} \in X$.
Der Nutzen von Spieler $i$ ist
\[
U_i(x^{(i)},x^{-i})
=
\alpha
\sum_{j\neq i}
f\!\bigl(d(x^{(i)},x^{(j)})\bigr)
\;+\;
\beta\, g\!\bigl(x^{(i)}\bigr),
\]
wobei
\begin{itemize}
\item $f$ eine fallende Funktion der Distanz ist
      (Kohärenzpräferenz),
\item $g$ eine individuelle Präferenzfunktion ist,
\item $\alpha,\beta>0$ die Gewichtung
      von Konsens vs.~Autonomie steuern.
\end{itemize}

Ein Spezialfall ist die konfliktbasierte Form
\[
f(d) = -C(x^{(i)},x^{(j)}),
\qquad
U_i = -\frac{1}{N-1}
\sum_{j\neq i} C(x^{(i)},x^{(j)}).
\]

\subsection*{4. p-adische Anpassungsdynamik (Best Response)}

Diskrete Zeit $t=0,1,2,\dots$.
Jeder Spieler passt ausschließlich die \emph{niedrigste Ebene} an,
auf der kein Konsens besteht.

Sei $k_i(t)$ die höchste Ebene,
auf der $x^{(i)}_t$ von der Mehrheitsentscheidung abweicht.
Dann gilt
\[
x^{(i)}_{t+1}
=
\operatorname*{arg\,max}_{y \in N_p^{(k_i(t))}(x^{(i)}_t)}
U_i\!\left(y, x^{-i}_t\right),
\]
wobei $N_p^{(k)}(x)$
die Menge der Alternativen bezeichnet,
die sich ausschließlich auf Ebene $k$
vom aktuellen Pfad unterscheiden.

Damit entstehen:
\begin{itemize}
\item stabile hohe Ebenen (träge Struktur),
\item flexible lokale Ebenen (Basisdynamik),
\item sprunghafte Phasenübergänge bei Konsensbruch oben.
\end{itemize}

\subsection*{5. Beispielinstanz}

Für $p=5$ und drei Ebenen
\[
k=0:\text{Dorf},\quad
k=1:\text{Bezirk},\quad
k=2:\text{Bundesland},
\]
mit Gewichten $w_0=1,\;w_1=3,\;w_2=9$
ergibt sich
\[
C(x,y)
=
\sum_{k=0}^2 w_k\,
\mathbf{1}[x_k\neq y_k],
\qquad
U_i = -\frac{1}{N-1}
\sum_{j\neq i} C(x^{(i)},x^{(j)}).
\]

\medskip
\noindent
\textbf{Interpretation:}
\[
\text{Kohärenz} \;\sim\; k(x,y)
\quad\text{(gemeinsame Konsenstiefe)};
\qquad
\text{Fragmentierung} \;\sim\; d(x,y)
\quad\text{(Ultrametrik-Abstand)}.
\]

