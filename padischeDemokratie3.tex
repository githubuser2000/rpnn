\section*{Erweiterter Entscheidungsraum mit drei Dimensionen}

\subsection*{1. Mehrdimensionaler p-adischer Entscheidungsraum}

Jede Entscheidung besitzt nun drei unabhängige Komponenten
\[
x = (x^{(S)},\,x^{(G)},\,x^{(V)}),
\]
wobei
\[
x^{(S)} = (x^{(S)}_0,x^{(S)}_1,\dots)
\quad\text{(Stärke / Kampfkraft)},
\]
\[
x^{(G)} = (x^{(G)}_0,x^{(G)}_1,\dots)
\quad\text{(Geldmenge / ökonomische Ressourcen)},
\]
\[
x^{(V)} = (x^{(V)}_0,x^{(V)}_1,\dots)
\quad\text{(Vote / Stimmwahl)}.
\]

Jede Komponente ist ein Pfad im Hierarchiebaum mit Ebenen
(Dorf, Bezirk, Bundesland, Nation, \dots).

\subsection*{2. p-adische Konsenstiefe je Dimension}

Für zwei Entscheidungen $x,y$ definieren wir die
dimensionale Konsenstiefe
\[
k_D(x,y)
=
\max\{\,k \mid x^{(D)}_j = y^{(D)}_j
\ \text{für alle } j \ge k\,\},
\qquad
D \in \{S,G,V\}.
\]

Daraus entstehen drei Ultrametriken
\[
d_D(x,y)=p^{-k_D(x,y)}.
\]

\subsection*{3. Gesamtmetrik (gewichtete politische Struktur)}

Die gesellschaftliche Distanz ergibt sich als
gewichtete Kombination der Dimensionen:
\[
d(x,y)
=
\lambda_S\, d_S(x,y)
\;+\;
\lambda_G\, d_G(x,y)
\;+\;
\lambda_V\, d_V(x,y),
\]
mit Gewichten
\[
\lambda_S,\lambda_G,\lambda_V > 0,
\qquad
\lambda_S+\lambda_G+\lambda_V = 1,
\]
die die Relevanz von
Kampfkraft, Geldmenge und Stimmwahl ausdrücken.

\subsection*{4. Konfliktkosten pro Dimension}

\[
C(x,y)
=
\sum_{D\in\{S,G,V\}}
\sum_{k\ge 0}
w_{D,k}\,
\mathbf{1}\!\bigl[
x^{(D)}_k \neq y^{(D)}_k
\bigr],
\]
mit hierarchie- und dimensionsspezifischen Gewichten
\[
w_{S,k}:\text{Macht-/Sicherheitskonflikte},\qquad
w_{G,k}:\text{ökonomische Konflikte},\qquad
w_{V,k}:\text{demokratische Abstimmungskonflikte}.
\]

\subsection*{5. Nutzenfunktion im dreidimensionalen Raum}

\[
U_i
=
\alpha
\sum_{j\neq i}
F\!\bigl(d(x^{(i)},x^{(j)})\bigr)
\;+\;
\beta_S\, g_S(x^{(i)})
\;+\;
\beta_G\, g_G(x^{(i)})
\;+\;
\beta_V\, g_V(x^{(i)}),
\]
wobei
\begin{itemize}
\item $g_S$ = strategischer Nutzen (Kampfkraft / Sicherheit),
\item $g_G$ = ökonomischer Nutzen (Ressourcen / Wohlstand),
\item $g_V$ = politischer Nutzen (Wahlpräferenz / Repräsentation),
\item $\beta_S,\beta_G,\beta_V>0$ die Prioritäten der Dimensionen steuern.
\end{itemize}

\subsection*{6. Interpretation}

\[
\text{Stabilität}
\;\sim\;
\text{gemeinsame Konsenstiefe in } (S,G,V),
\qquad
\text{Machtverschiebung}
\;\sim\;
\text{Bruch auf hoher Ebene in } S,
\]
\[
\text{ökonomische Fragmentierung}
\;\sim\;
\text{Bruch in } G,
\qquad
\text{demokratische Polarisierung}
\;\sim\;
\text{Bruch in } V.
\]

