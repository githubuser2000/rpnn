\documentclass[a4paper,12pt]{article}
\usepackage[utf8]{inputenc}
\usepackage{amsmath, amssymb}

\title{Umwandlung ganzer Zahlen in perfektoide Zahlen über p-adische Zahlen}
\author{}
\date{}

\begin{document}
\maketitle

\section*{Einleitung}
Wir betrachten eine Menge ganzer Zahlen 
\[
\{ n_1, n_2, \dots, n_k \} \subset \mathbb{Z},
\] 
die wir in \emph{perfektoide Zahlen} überführen möchten. 
Dazu verwenden wir klassische Werkzeuge der Zahlentheorie wie Primfaktorzerlegung, 
den größten gemeinsamen Teiler (ggT) und das kleinste gemeinsame Vielfache (kgV).

\section*{1. Primfaktorzerlegung}
Jede Zahl \(n_i\) wird in Primfaktoren zerlegt:
\[
n_i = \prod_{j} p_j^{\alpha_{i,j}}, \quad p_j \text{ Primzahl}, \quad \alpha_{i,j} \in \mathbb{Z}_{\ge 0}.
\]
Damit haben wir jede Zahl als Produkt von Potenzen verschiedener Primzahlen geschrieben, 
was die Grundlage für die p-adische Betrachtung bildet.

\section*{2. Größter gemeinsamer Teiler und kleinstes gemeinsames Vielfaches}
\subsection*{Größter gemeinsamer Teiler (ggT)}
\[
\gcd(n_1, n_2, \dots, n_k) = \prod_j p_j^{\min(\alpha_{1,j}, \dots, \alpha_{k,j})}.
\]
Der ggT entspricht den minimalen Exponenten jeder Primzahl, die in allen Zahlen vorkommen.

\subsection*{Kleinstes gemeinsames Vielfaches (kgV)}
\[
\mathrm{lcm}(n_1, n_2, \dots, n_k) = \prod_j p_j^{\max(\alpha_{1,j}, \dots, \alpha_{k,j})}.
\]
Das kgV entspricht den maximalen Exponenten jeder Primzahl, die in mindestens einer Zahl vorkommen.

\section*{3. Überführung in p-adische Zahlen}
Für jede Primzahl \(p_j\) definieren wir eine p-adische Zahl:
\[
n_i \mapsto n_i^{(p_j)} \in \mathbb{Z}_{p_j}, \quad
n_i^{(p_j)} = \sum_{k \ge 0} a_k p_j^k, \quad a_k \in \{0,1,\dots,p_j-1\}.
\]
\begin{itemize}
    \item Der ggT entspricht den minimalen Exponentenwerten.
    \item Das kgV entspricht den maximalen Exponentenwerten.
\end{itemize}

\section*{4. Bildung der perfektoiden Zahl}
Perfektoide Zahlen sind das Projektivlimit der p-adischen Zahlen über alle Primzahlen:
\[
\mathbb{Z}^{\mathrm{perf}} \simeq \varprojlim_{p} \mathbb{Z}_p.
\]

Daher können wir aus der Menge \(\{n_i\}\) eine perfektoide Zahl konstruieren:
\[
\mathbf{n} = (n_1^{(p_1)}, n_1^{(p_2)}, \dots) \in \prod_p \mathbb{Z}_p \longrightarrow \mathbb{Z}^{\mathrm{perf}}.
\]
Die ggT-Exponenten liefern die minimalen Bausteine, 
die kgV-Exponenten die maximalen, und alles dazwischen wird über das p-adische System konsistent dargestellt.

\section*{5. Zusammenfassung}
\begin{enumerate}
    \item Zerlege die ganzen Zahlen in Primfaktoren.  
    \item Bestimme für jede Primzahl die minimalen (ggT) und maximalen (kgV) Exponenten.  
    \item Wandle jede Zahl in jede \(\mathbb{Z}_p\) um.  
    \item Setze die p-adischen Komponenten als Projektivsystem zusammen, um die perfektoide Zahl zu erhalten.
\end{enumerate}

\section*{Mathematische Kompaktdarstellung}
\[
\begin{aligned}
n_i &= \prod_{j} p_j^{\alpha_{i,j}}, \\
\gcd(n_1, \dots, n_k) &= \prod_j p_j^{\min(\alpha_{1,j}, \dots, \alpha_{k,j})}, \\
\mathrm{lcm}(n_1, \dots, n_k) &= \prod_j p_j^{\max(\alpha_{1,j}, \dots, \alpha_{k,j})}, \\
\mathbf{n} &= \varprojlim_p n_i^{(p)} \in \mathbb{Z}^{\mathrm{perf}}.
\end{aligned}
\]

\end{document}

